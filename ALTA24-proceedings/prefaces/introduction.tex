\hyphenation{ALTA} % to avoid breaking ALTA over two lines

Welcome to the \textbf{22nd Annual Workshop of the Australasian Language Technology Association (ALTA 2024)}. Hosted on the Acton campus of the Australian National University in Canberra, ALTA 2024 will provide a platform for the exchange of ideas, exploration of innovations, and discussion of the latest advancements in language technology. The conference acknowledges the significance of its location on the \textit{traditional lands of the Ngunnawal and Ngambri peoples}, underscoring a commitment to inclusivity and respect.\newline

ALTA 2024 convenes leading researchers, industry experts, and practitioners in the fields of natural language processing (NLP) and computational linguistics. This year, ALTA will focus on the critical role of large language models (LLMs) in shaping contemporary research and industrial applications.\newline %Through keynote addresses and discussions, the program will critically evaluate LLMs’ capabilities, limitations, and broader societal impact.\newline

ALTA has seen a remarkable growth in 2024. We received 43 submissions, a 1.79 times increase from the 24 submissions in 2023. This trajectory aligns with trends observed in global NLP research communities such as ACL and EMNLP. Following a rigorous and competitive review process, 21 submissions were accepted, comprising 10 long papers, 6 short papers, and 5 abstracts (not included in proceedings). The acceptance rate for papers included in the proceedings is 37.21\% (16/43), reflecting a more selective process compared to 2023’s 66.67\% acceptance rate (16 out of 24 papers). We are also delighted to observe an increase in international participation. Of the accepted submissions, 85.71\% (18 submissions) originate from Australia, 9.52\% (2) from the USA, and 4.76\% (1) from Malaysia.\newline

This year's submissions showcase advancements across a wide array of topics. From educational applications such as personalised tutoring systems to healthcare-focused advancements like dementia self-disclosure detection and synthetic clinical text generation, the accepted papers demonstrate the versatility of NLP technologies. There is an evident focus on low-resource language processing, multilingual NLP, and domain-specific applications, with papers exploring practical solutions for real-world problems such as hate speech detection and legal document processing. A clear emphasis is given to bridging the gap between research and application. The focus on small-scale LLMs resonates with the community's efforts to develop resource-efficient and accessible AI systems.\newline 

We want to sincerely thank everyone who helped make ALTA 2024 a reality. A special thank you to our keynote speakers for fantastic presentations: Prof. Eduard Hovy (University of Melbourne), Prof. Jing Jiang (Australian National University), Prof. Steven Bird (Charles Darwin University), and Kyla Quinn (Australian Department of Defence). Thank you to the members of the discussion panel for an insightful conversation: Kyla Quinn, Prof. Hanna Suominen (Australian National University), and Luiz Pizzato (Commonwealth Bank). Thank you to the members of organising committee and volunteers for their hard work in preparing and running ALTA. We extend our heartfelt appreciation to the reviewers: your diligence and insightful feedback played an integral role in upholding the quality and rigor of the review process. Lastly, ALTA 2024 gratefully acknowledges the support of our sponsors: Defence Science and Technology Group (Platinum), Google (Gold), ARDC (Silver), and Commonwealth Bank, University of Melbourne, and Unsloth AI (Bronze). We are also proud to have The Australian National University as our host. The success of this workshop would not be possible without your invaluable contributions.\newline

Welcome to ANU and Canberra! We hope that you enjoy ALTA 2024, and look forward to a rewarding and inspiring time together.\newline

\hspace*{\fill}Tim Baldwin\newline
\hspace*{\fill}Sergio José Rodríguez Méndez\newline
\hspace*{\fill}Nicholas I-Hsien Kuo\newline
\hspace*{\fill}\textit{ALTA 2024 Program Chairs}
